
% $Id$

\documentclass{report}
%\documentclass[draft]{report}

\usepackage{a4}
\usepackage{shortvrb}
\usepackage{pslatex}

\usepackage{verbatim}
\usepackage{alltt}		% as verbatim, but interpret \ { }

\newenvironment{code}{\verbatim}{\endverbatim}
\newenvironment{codem}{\alltt}{\endalltt}	%interpret \
\newcommand{\ttt}[1]{{\tt #1}}

\newcommand{\shy}{{\tt SHYFEM}}
\newcommand{\psp}{{\tt SHYFEM}}

\input{P_version.tex}
\newcommand{\shyname}{shyfem-\version}
\newcommand{\shydist}{\shyname.tar.gz}
\newcommand{\basedir}{/home/model}
\newcommand{\shydir}{\basedir/shyfem-\version}

\newcommand{\descrpsep}{\vspace{0.2cm}}
\newcommand{\descrpitem}[1]{\descrpsep\parbox[t]{2cm}{#1}}
\newcommand{\descrptext}[1]{\parbox[t]{8cm}{#1}\descrpsep}
\newcommand{\descrp}[1]{\descrpsep\parbox[t]{4cm}{#1}}
\newcommand{\descrpn}[1]{\parbox[t]{4cm}{#1}}

\newcommand{\densityunit}{kg\,m${}^{-3}$}
\newcommand{\accelunit}{m\,s${}^{-2}$}
\newcommand{\maccelunit}{m${}^{4}$\,s${}^{-2}$}
\newcommand{\dischargeunit}{m${}^3$\,s${}^{-1}$}

\newcommand{\ten}[1]{$\cdot 10^{#1}$}
\newcommand{\degrees}{${}^o$}

\newcommand{\figref}[1]{\ref{fig:#1}}
\newcommand{\Fig}{Fig.~}

\newcommand{\todo}[1]{This section still has to be written by #1}

\parindent 0cm

\newcommand{\importstr}[2]{%
\begin{figure}
\begin{alltt}
\input{#1.str}
\end{alltt}
\caption{#2}
\label{fig:#1}
\end{figure}
}

\MakeShortVerb{\|}

\hyphenation{ ba-thy-me-try stop-ped sim-u-la-ted }

%%%%%%%%%%%%%%%%%%%%%%%%%%%%%%%%%%%%%%%%%%%%%%%%%%%%%%%%% front matter

\title{%
	\shy{} 
	\\Finite Element Model for Coastal Seas
	\\~
	\\User Manual
	}

\author{%
	The SHYFEM Group
	\\Georg Umgiesser
	\\Oceanography, ISMAR-CNR
	\\Arsenale Tesa 104, Castello 2737/F
	\\30122 Venezia, Italy
	\vspace{0.5cm}
	\\georg.umgiesser@ismar.cnr.it
	\vspace{1cm}
	\\Version \VERSION
	}

%\address{ISDGM-CNR}

%\date{}		%uncomment for no date

%%%%%%%%%%%%%%%%%%%%%%%%%%%%%%%%%%%%%%%%%%%%%%%%%%%%%%%%% document

\begin{document}

\bibliographystyle{plain}

\pagenumbering{roman}
\pagestyle{plain}

\maketitle

%\begin{abstract}
% to be written
%\end{abstract}

\thispagestyle{empty}

\newpage
\tableofcontents
\newpage

%%%%%%%%%%%%%%%%%%%%%%%%%%%%%%%%%%%%%%%%%%%%%%%%%%%%%%%%%%%%%%%%%%%%%%%%
%%%%%%%%%%%%%%%%%%%%%%%%%%%%%%%%%%%%%%%%%%%%%%%%%%%%%%%%%%%%%%%%%%%%%%%%
%%%%%%%%%%%%%%%%%%%%%%%%%%%%%%%%%%%%%%%%%%%%%%%%%%%%%%%%%%%%%%%%%%%%%%%%

\chapter*{Disclaimer}
	\addcontentsline{toc}{chapter}{Disclaimer}

	\input{disclaimer}

\newpage
\pagenumbering{arabic}

%%%%%%%%%%%%%%%%%%%%%%%%%%%%%%%%%%%%%%%%%%%%%%%%%%%%%%%%%%%%%%%%%%%%%%%%
%%%%%%%%%%%%%%%%%%%%%%%%%%%%%%%%%%%%%%%%%%%%%%%%%%%%%%%%%%%%%%%%%%%%%%%%
%%%%%%%%%%%%%%%%%%%%%%%%%%%%%%%%%%%%%%%%%%%%%%%%%%%%%%%%%%%%%%%%%%%%%%%%

% external software
% contributors

\chapter{Overview}

	\section{What is it}
	\input{over_what}

	%\section{Why use it}
	%\todo{Georg}

	%\section{How to use it}
	%\todo{Georg}

	%\section{Where to get it}
	%\todo{Georg}

	%\section{Features}
	%\todo{Georg}

	%\section{License}
	%\todo{Georg}

%%%%%%%%%%%%%%%%%%%%%%%%%%%%%%%%%%%%%%%%%%%%%%%%%%%%%%%%%%%%%%%%%%%%%%%%
%%%%%%%%%%%%%%%%%%%%%%%%%%%%%%%%%%%%%%%%%%%%%%%%%%%%%%%%%%%%%%%%%%%%%%%%
%%%%%%%%%%%%%%%%%%%%%%%%%%%%%%%%%%%%%%%%%%%%%%%%%%%%%%%%%%%%%%%%%%%%%%%%

\chapter{Installing SHYFEM}

	\section{Downloading and unpacking}
	\input{download}

	\section{Needed software}
	\input{software}

	\section{Installation}
	\input{installation}

	\section{Compilation}
	\input{compilation}

	\section{Compatibility problems}
	\input{compatibility}

	\section{Compilation options}
	\input{summary_installing}

%%%%%%%%%%%%%%%%%%%%%%%%%%%%%%%%%%%%%%%%%%%%%%%%%%%%%%%%%%%%%%%%%%%%%%%%
%%%%%%%%%%%%%%%%%%%%%%%%%%%%%%%%%%%%%%%%%%%%%%%%%%%%%%%%%%%%%%%%%%%%%%%%
%%%%%%%%%%%%%%%%%%%%%%%%%%%%%%%%%%%%%%%%%%%%%%%%%%%%%%%%%%%%%%%%%%%%%%%%

\chapter{Preprocessing: The numerical grid}

	\section{Overview}
	\input{grid_overview}

	\section{Coastline and bathymetry}
	\input{grid_convert}

	\section{Boundary line: smoothing and reducing}
	\input{grid_reduce}

%	\section{Other information}
%	\input{grid_else}

	\section{Manipulating nodes, lines and elements: the grid program}
	\input{grid_prg}

	\section{Creating the mesh}
	\input{creating_mesh}

	\section{Interpolating the bathymetry into the grd file}
	\input{bathy_interp}

	\section{Creating the bas file}
	\input{pre_basin}

%%%%%%%%%%%%%%%%%%%%%%%%%%%%%%%%%%%%%%%%%%%%%%%%%%%%%%%%%%%%%%%%%%%%%%%%
%%%%%%%%%%%%%%%%%%%%%%%%%%%%%%%%%%%%%%%%%%%%%%%%%%%%%%%%%%%%%%%%%%%%%%%%
%%%%%%%%%%%%%%%%%%%%%%%%%%%%%%%%%%%%%%%%%%%%%%%%%%%%%%%%%%%%%%%%%%%%%%%%

\chapter{Running SHYFEM}

	\input{running_shyfem}

	\section{The parameter input file (str)}
	\input{model_how_to_run}

%	\section{Adjusting to the basin}
%	\todo{Georg}

%	\section{Basic file formats}
%	\todo{Georg}

	\section{Basic usage}

		\input{basic_usage}

		\subsection{Minimal simulation}
		\input{basic_minimal}

%		\input{usage.tex}	%??

		\subsection{Boundary conditions}
		\input{bound_cond}

%		\subsection{Writing output}
%		\todo{Georg}

%		\subsection{Simulating passive tracers}
%		\todo{Georg}

%		\subsection{Temperature and salinity}
%		\todo{Georg}

%		\subsection{Linear and non-linear simulations}
%		\todo{Georg}

%		\subsection{Coriolis force}
%		\todo{Georg}

		\subsection{Wind forcing}
		\input{wind_forcing}

	\section{Advanced usage}

		\subsection{Variable time step}
		\input{var_time_step}

		\subsection{3D computations}
		
%------------------------------------------------------------------------
%
%    Copyright (C) 1985-2020  Georg Umgiesser
%
%    This file is part of SHYFEM.
%
%    SHYFEM is free software: you can redistribute it and/or modify
%    it under the terms of the GNU General Public License as published by
%    the Free Software Foundation, either version 3 of the License, or
%    (at your option) any later version.
%
%    SHYFEM is distributed in the hope that it will be useful,
%    but WITHOUT ANY WARRANTY; without even the implied warranty of
%    MERCHANTABILITY or FITNESS FOR A PARTICULAR PURPOSE. See the
%    GNU General Public License for more details.
%
%    You should have received a copy of the GNU General Public License
%    along with SHYFEM. Please see the file COPYING in the main directory.
%    If not, see <http://www.gnu.org/licenses/>.
%
%    Contributions to this file can be found below in the revision log.
%
%------------------------------------------------------------------------

\subsubsection{General information}

The basic way to run the model is in 2D, computing for each element of the
grid one value for the whole water column.  All the variables are computed
in the center of the layer, halfway down the total depth.  Deeper basins
or highly variable bathymetry can require for the correct reproduction
of the velocities, temperature and salinity the need for 3D computation.

The 3D computation is performed with $z-$layers, sigma layers or with
a hybrid formulation.  In the default $z-$layer formulation, each layer
horizontally has constant depth over the whole basin, but vertically
the layer thickness may vary between different layers. However, the
first layer (surface layer) is of varying thickness because of the water
level variation, and the last layer of an element might be only partially
present due to the bathymetry.

Layers are counted from the the surface layer (layer 1) down to the
maximum layer, depending again on the local depth. Therefore, elements
(and nodes) normally have a different total number of layers from one to
each other. This is opposed to sigma layers where the number of total
layers is constant all over the basin, but the thickness of each layer
varies between different elements.

\subsubsection{$z-$layers}

In order to use $z-$layers for 3D computations a new section |$layers|
has to be introduced into the |STR| file, where the sequence of depth
values of the bottom of the layers has to be declared.  Layer
depths must be declared in increasing order. An example of a |$layer|
section is given in figure \figref{zlayers}. Please note that the maximum
depth of the basin in the example must not exceed 20 m.

\begin{figure}[ht]
\begin{verbatim}
$layers
	2 4 6 8 10 13 16 20
$end
\end{verbatim}
\caption{Example of section {\tt \$layers} for z layers. The maximum depth of 
the basin is 20 meters. The first 5 layers have constant thickness 
of 2 m, while the last three vary between 3 and 4 m.}
\label{fig:zlayers}
\end{figure}

A specific treatment for the bottom layer has to be carried out.  In fact,
if the model runs on basins with variable bathymetry, for each element
there will be a different total number of layers. The bathymetric value
normally does not coincide with one of the layer depths, and therefore
the last layer must be treated separately.

To declare how to treat the last layer two parameters have to be
inserted in the |$para| section. The first is |hlvmin|, the minimum
depth, expressed as a percentage with respect to the full layer depth,
ranging between 0 and 1, This is the fraction that the last layer
must have in order to be maintained as a separate layer.  The second
parameter is |ilytyp| and it defines the kind of adjustment done on the
last layer. If it is set to 0 no adjustment is done, if it is set to 1
the depth of the last layer is adjusted to the one declared in the |STR|
file (full layer change).  If it is 2 the adjustment to the previous layer
is done only if the fraction of the last layer is smaller than |hlvmin|
(change of depth).  If it is 3 (default) the bathymetric depth is kept
and added to the last but one layer. Therefore with a value of 0 or 3
the total depth will never be changed, whereas with the other levels
the total depth might be adjusted.

As an example, take the layer definition of \Fig\figref{zlayers}. Let |hlvmin|
be set to 0.5, and let an element have a depth of 6.5 m. The total number
of layers is 4, where the first 3 have each a thickness of 2 m and the
last layer of this element (layer 4) is 0.5 m. However, the nominal
thickness of layer 4 is 2 m and therefore its relative thickness is 0.25
which is smaller than |hlvmin|. With |ilytyp|=0 no adjustment will be
done and the total number of layers in this element will be 4 and the
last layer will have a thickness of 0.5 m.  With |ilytyp|=1 the total
number of layers will be changed to 3 (all of them with 2 m thickness)
and the total depth will be adjusted to 6 m. The same will happen with
|ilytyp|=2, because the relative thickness in layer 4 is smaller than
|hlvmin|.  Finally, with |ilytyp|=3 the total number of layers will be
changed to 3 but the remaining depth of 0.5 m will be added to layer 3
that will become 2.5 m.

In the case the element has a depth of 7.5 m, the relative thickness is
now 0.75 and greater than |hlvmin|.  In this case, with |ilytyp|=0, 2
and 3 no adjustment will be done and the total number of layers in this
element will be 4 and the last layer will have a thickness of 1.5 m.
With |ilytyp|=1 the total number of layers will be kept as 4 but the
total depth will be adjusted to 8 m. This will make all layers equal to
2 m thickness.

A specific treatment for the surface layer is also needed. Standard 
$z-$layers are coded with the first layer of variable thickness that, 
however, must never become dry. This is the default usage of $z-$layers.
As an alternative, the $z$-star layers can be used. You need to specify in the 
|$para| section the parameter |nzadapt| equal or greater to
the maximum number of layers. For the previous example
one should set |nzadapt|$\ge 8$. If one wants to use fixed
interface for the interior part of the water column, there is also the
possibility to move only the surface layers with a $z-$star type 
transformation. This reminds of $z$-star over $z-$layers. To use
this slicing, you need to set |nzadapt|, the minimum number of moving layers 
(when the water level moves downward). For example |nzadapt|$=2$ means that, 
at minimum, two layers will absorb a downward motion of the water level. 
Please note that this feature is still experimental.


\subsubsection{Sigma layers}

Sigma layers use a constant number of layers all over the basin. They
are easier to use than z layers, because only one parameter has to
be specified. In the |$para| section of the |STR| file 
the parameter |nsigma| has to be set
to the number of desired sigma layers. The layers are then equally spaced
between each other.

Sigma layers can be also specified in the |$layers| section. In this case
the negative percentage of the layers have to be given.
An example is given in figure \figref{slayers}. This is only useful if
the layers are not equally spaced, because for equally spaced sigma layers
the parameter |nsigma| in the |STR| can be used.

In the bathymetry file depth values have to be given on nodes and not
on elements. in case the utility |shybas| can be used to convert between
elemental and nodal depth values.

\begin{figure}[ht]
\begin{verbatim}
$layers
	-0.2 -0.4 -0.6 -0.8 -1.0
$end
\end{verbatim}
\caption{Example of section {\tt \$layers} for sigma layers.
The depth is divided in 5 equally spaced layers. Please note that
this division could have also been achieved setting {\tt nsigma} to the
value of 5.}
\label{fig:slayers}
\end{figure}

\subsubsection{Hybrid layers}

Hybrid layers are also called "sigma over zeta" layers. They are a
mixture between sigma layers close to the surface and zeta layers in
the deeper parts of the basin. This is useful if strong bathymetry
gradients are present. This avoids possible instabilities due to the
sigma layers in the deeper parts.

For the hybrid layers a depth of closure has to be defined. This is the
depth value above which sigma layers are used and below which zeta levels
are applied. Please note that the basin has to be prepared in order that
depth values are given on nodes and in an elements the three depth values
on the vertices are either higher equal or lower equal than the depth
of closure. The routine |shybas| can be used in order to create
a grid compatible with hybrid coordinates. An example of how
to specify hybrid layers is given in
figure \figref{hlayers}.

\begin{figure}[ht]
\begin{verbatim}
$layers
	-0.2 -0.4 -0.6 -0.8 10. 20. 30. 40. 50
$end
\end{verbatim}
\caption{Example of section {\tt \$layers} for hybrid layers.
The depth is divided in 5 equally sigma layers on the surface above 10 meters
(which is the depth of closure) and zeta layers below until a depth
of 50 meters.}
\label{fig:hlayers}
\end{figure}

Please note that hybrid layers are still experimental. So use with care.




\subsubsection{Vertical viscosity}

The introduction of layers requires also to define the values of
vertical eddy viscosity and eddy diffusivity.  In any case a value of
these two parameters has to be set if the 3D run is performed. This
could be done by setting a constant value of the parameters |vistur|
(vertical viscosity) and |diftur| (vertical diffusivity). In this case
possible values are between 1\ten{-2} and 1\ten{-5}, depending on the
stability of the water column. Higher values (1\ten{-2}) indicate higher
stability and a stronger barotropic behavior.

The other possibility is to compute the vertical eddy coefficients through
a turbulence closure scheme. This usage will be described in the section
on turbulence.



		\subsection{Baroclinic terms}
		\input{barocl}

		\subsection{Restart}
		\input{restart}

		\subsection{Turbulence}
		\input{turbulence}

		\subsection{Sediment transport}
		\input{sediment}

		\subsection{Wind waves}
		\input{S_wave}

		\subsection{Tidal potential}
		\input{S_tidef}

		\subsection{Meteo forcing}
		\input{meteo_forcing}

		\subsection{Lagrangian particle module}
		Lagrangian analysis provides a powerful tool to evaluate the output
of ocean circulation models. SHYFEM is equipped with a 3-D 
particle-tracking module, which simulates the trajectory of 
particles as a function of the hydrodynamics. 

The vertical components of the turbulent diffusion velocity is
computed using the Milstein scheme reported by Grawe 
\cite{Grawe2010}. The horizontal diffusion was computed using a 
random walk technique based on Fisher \cite{Fisher1979}, with the 
turbulent diffusion coefficients obtained by means of the 
Smagorinsky \cite{Smagorinsky1993} formulation. The wind drag 
and Stokes drift contribution to the total transport is parametrized
by |stkpar| factor. An additional calibration parameter to account
for the drifter inertia could be set (|dripar|). The model allows
particle to beach on the shore (|lbeach|). 

The particle-tracking model can be also used off-line (parameter
|idtoff|). In this case it uses the Eulerian hydrodynamic fields 
generated by the forecast system. The main advantage of the 
off-line approach is that the trajectory calculation typically 
takes much less computational effort than the driving hydrodynamic 
model.

The lagrangian particles can be released:
\begin{itemize}
\item inside the given areas (filename |lgrlin|). If this file is not 
      specified they are released over the whole domain. The amount of
      particles released and the time step is specified by |nbdy| and 
      |idtl|.
\item at selected times and location, e.g. along a drifter track
      (filename |lgrtrj|). |nbdy| particles are released at the times
      and location specified in the file.
\item as initial particle distribution (filename |lgrini|) at time
      |itlgin|. This file has the same format as the lagrangian output.
\item at the open boundaries, either as particles per second or per
      volume flux (parameter |lgrpps|).
\end{itemize}

The particle-tracking model is activated by setting |ilagr| > 0.
The lagrangian module runs between the times |itlanf| and |itlend|.
See more details in the list of parameters and they description 
reported in the appendix.

The lagrangian model can be used to specifically simulate sediments
(|ised=1|), oil (|ioil=1|) and larvae (|ilarv=1|).


%		\subsection{Large grids and projections}
%		\todo{Debora}

%%%%%%%%%%%%%%%%%%%%%%%%%%%%%%%%%%%%%%%%%%%%%%%%%%%%%%%%%%%%%%%%%%%%%%%%
%%%%%%%%%%%%%%%%%%%%%%%%%%%%%%%%%%%%%%%%%%%%%%%%%%%%%%%%%%%%%%%%%%%%%%%%
%%%%%%%%%%%%%%%%%%%%%%%%%%%%%%%%%%%%%%%%%%%%%%%%%%%%%%%%%%%%%%%%%%%%%%%%

%\chapter{Other modules}

%	\section{Residence times}
%	\todo{Andrea}

%	\section{Ecological module (EUTRO)}
%	\todo{Michol (c'e' gia' una descrizione di Donata)}

%	\input{bio3d.tex}
%	\subsection{Parameters for the Water Quality Module}
%	\input{S_biopar_h.tex}

%%%%%%%%%%%%%%%%%%%%%%%%%%%%%%%%%%%%%%%%%%%%%%%%%%%%%%%%%%%%%%%%%%%%%%%%
%%%%%%%%%%%%%%%%%%%%%%%%%%%%%%%%%%%%%%%%%%%%%%%%%%%%%%%%%%%%%%%%%%%%%%%%
%%%%%%%%%%%%%%%%%%%%%%%%%%%%%%%%%%%%%%%%%%%%%%%%%%%%%%%%%%%%%%%%%%%%%%%%

%\chapter{Postprocessing}
%
% Da qui in poi ci pensiamo piu' tardi...
%
%\section{Running the post processing routines}
%\section{Time series: gnuplot}
%\section{Plotting spatial data: plots}
%	\subsection{Basic usage}
%	\subsection{Advanced usage}
%\section{Exporting data}

%%%%%%%%%%%%%%%%%%%%%%%%%%%%%%%%%%%%%%%%%%%%%%%%%%%%%%%%%%%%%%%%%%%%%%%%
%%%%%%%%%%%%%%%%%%%%%%%%%%%%%%%%%%%%%%%%%%%%%%%%%%%%%%%%%%%%%%%%%%%%%%%%
%%%%%%%%%%%%%%%%%%%%%%%%%%%%%%%%%%%%%%%%%%%%%%%%%%%%%%%%%%%%%%%%%%%%%%%%

%\chapter{Final thoughts}

%%%%%%%%%%%%%%%%%%%%%%%%%%%%%%%%%%%%%%%%%%%%%%%%%%%%%%%%%%%%%%%%%%%%%%%%
%%%%%%%%%%%%%%%%%%%%%%%%%%%%%%%%%%%%%%%%%%%%%%%%%%%%%%%%%%%%%%%%%%%%%%%%
%%%%%%%%%%%%%%%%%%%%%%%%%%%%%%%%%%%%%%%%%%%%%%%%%%%%%%%%%%%%%%%%%%%%%%%%
%%%%%%%%%%%%%%%%%%%%%%%%%%%%%%%%%%%%%%%%%%%%%%%%%%%%%%%%%%%%%%%%%%%%%%%%
%%%%%%%%%%%%%%%%%%%%%%%%%%%%%%%%%%%%%%%%%%%%%%%%%%%%%%%%%%%%%%%%%%%%%%%%



\appendix


\chapter{Hydrodynamic equations and resolution techniques}

	\input{equat.tex}

\chapter{File formats}
\label{file_formats}

%	\section{STR file}

	\section{GRD file}
	\input{grd_format}

%	\section{Time series}

%	\section{Output file formats}





\chapter{Parameter list}


\section{Parameter list for the SHYFEM model}


\subsection{Section {\tt \$title}}

This section must be always the first section in the parameter input file.
It contains only three lines. An example is given in 
figure \ref{fig:titleexample}.

\begin{figure}[ht]
\begin{verbatim}
$title
        free one line description of simulation
        name_of_simulation
        name_of_basin
$end
\end{verbatim}
\caption{Example of section {\tt \$title}}
\label{fig:titleexample}
\end{figure}

The first line of this section is a free one line description of
the simulation that is to be carried out. The next line contains
the name of the simulation.
All created files will use this name in the main part of the file name
with different extensions. Therefore the hydrodynamic output file
(extension |out|) will be named |name_of_simulation.out|.
The last line gives the name of the basin file to be used. This
is the pre-processed file of the basin with extension |bas|.
In our example the basin file |name_of_basin.bas| is used.

The directory where this files are read from or written to depends
on the settings in section {\tt \$name}. Using the default
the program will read from and write to the current directory.

\subsection{Section {\tt \$para}}

This section defines the general behavior of the simulation,
gives various constants of parameters and determines what
output files are written. In the following the meaning of
all possible parameters is given.

Note that the only compulsory parameters in this section are 
the ones that chose the duration of the simulation and the
integration time step. All other parameters are optional.

\input{S_para_h.tex}

\subsection{Section {\tt \$proj}}
\input{P_proj.tex}

\subsection{Section {\tt \$waves}}
\input{P_wave.tex}

\subsection{Section {\tt \$sedtr}}
\input{P_sediment.tex}

\subsection{Section {\tt \$wrt}}
\input{P_wrt.tex}

\subsection{Section {\tt \$lagrg}}
\input{P_lagrg.tex}

\subsection{Section {\tt \$name}}

In this sections names of directories or input files can be
given. All directories default to the current directory,
whereas all file names are empty, i.e., no input files are
given.

\input{S_name.tex}
\input{S_name_h.tex}

\subsection{Section {\tt \$bound}}

\input{S_bound.tex}


\subsection{Section {\tt \$wind}}

\input{S_wind.tex}



\subsection{Section {\tt \$extra}}

In this section the node numbers of so called ``extra'' points are given. 
These are points where water level and velocities are written to create
a time series that can be elaborated later. The output for these ``extra''
points consumes little memory and can be therefore written with a
much higher frequency (typically the same as the integration time step)
than the complete hydrodynamic output. The output is written
to file EXT.

The node numbers are specified in a free format on one ore more lines.
An example can be seen in figure \ref{fig:example}. No keywords
are expected in this section.


\subsection{Section {\tt \$flux}}

In this section transects are specified through which the discharge
of water is computed by the program and written to file FLX.
The transects are defined by their nodes through which they run.
All nodes in one transect must be adjacent, i.e., they must form a
continuous line in the FEM network.

The nodes of the transects are specified in free format. Between
two transects one or more 0's must be inserted. An example is given in
figure \ref{fig:fluxexample}.

\begin{figure}[ht]
\begin{verbatim}
$flux
	1001 1002 1004 0
	35 37 46 0 0 56 57 58 0
	407
	301
	435 0 89 87
$end
\end{verbatim}
\caption{Example of section {\tt \$flux}}
\label{fig:fluxexample}
\end{figure}

The example shows the definition of 5 transects. As can be seen, the 
nodes of the transects can be given on one line alone (first transect),
two transects on one line (transect 2 and 3), spread over more lines
(transect 4) and a last transect.



\section{Parameter list for the post processing routines}

The format of the parameter input file is the same as the one for
the main routine. Please see this section for more information
on the format of the parameter input file.

Some sections of the parameter input file are identical to the 
sections used in the main routine. For easier reference we will
repeat the possible parameters of these section here.






\subsection{Section {\tt \$title}}

This section must be always the first section in the parameter input file.
It contains only three lines. An example has been given already in 
figure \ref{fig:titleexample}.

The only difference with respect to the {\tt \$para} section of the main
routine is the first line. Here any description of the output can be used.
It is just a way to label the parameter file.  The other two line with
the name of simulation and the basin are used to open the files needed
for plotting.


\subsection{Section {\tt \$para}}

\input{S_para_a.tex}


\subsection{Section {\tt \$color}}

\input{S_color.tex}


\subsection{Section {\tt \$arrow}}

\input{S_arrow.tex}


\subsection{Section {\tt \$legend}}

\input{S_legend.tex}


\subsection{Section {\tt \$legvar}}

\input{S_legvar.tex}


\subsection{Section {\tt \$name}}

\input{S_name.tex}







\bibliography{abbrev,lag,sedi,lagra}
\addcontentsline{toc}{chapter}{Bibliography}



\end{document}




