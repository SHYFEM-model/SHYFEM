
%------------------------------------------------------------------------
%
%    Copyright (C) 1985-2020  Georg Umgiesser
%
%    This file is part of SHYFEM.
%
%    SHYFEM is free software: you can redistribute it and/or modify
%    it under the terms of the GNU General Public License as published by
%    the Free Software Foundation, either version 3 of the License, or
%    (at your option) any later version.
%
%    SHYFEM is distributed in the hope that it will be useful,
%    but WITHOUT ANY WARRANTY; without even the implied warranty of
%    MERCHANTABILITY or FITNESS FOR A PARTICULAR PURPOSE. See the
%    GNU General Public License for more details.
%
%    You should have received a copy of the GNU General Public License
%    along with SHYFEM. Please see the file COPYING in the main directory.
%    If not, see <http://www.gnu.org/licenses/>.
%
%    Contributions to this file can be found below in the revision log.
%
%------------------------------------------------------------------------

% $Id$

\documentclass{report}
%\documentclass[draft]{report}

\usepackage{a4}
\usepackage{shortvrb}
\usepackage{pslatex}

\usepackage{verbatim}
\usepackage{alltt}		% as verbatim, but interpret \ { }

\newenvironment{code}{\verbatim}{\endverbatim}
\newenvironment{codem}{\alltt}{\endalltt}	%interpret \
\newcommand{\ttt}[1]{{\tt #1}}

\newcommand{\shy}{{\tt SHYFEM}}
\newcommand{\shyfem}{{\tt SHYFEM}}
\newcommand{\psp}{{\tt SHYFEM}}

\input{P_version.tex}
\newcommand{\shyname}{shyfem-\version}
\newcommand{\shydist}{\shyname.tar.gz}
\newcommand{\basedir}{/home/model}
\newcommand{\shydir}{\basedir/shyfem-\version}

\newcommand{\descrpsep}{\vspace{0.2cm}}
\newcommand{\descrpitem}[1]{\descrpsep\parbox[t]{2cm}{#1}}
\newcommand{\descrptext}[1]{\parbox[t]{8cm}{#1}\descrpsep}
\newcommand{\descrp}[1]{\descrpsep\parbox[t]{4cm}{#1}}
\newcommand{\descrpn}[1]{\parbox[t]{4cm}{#1}}

\newcommand{\densityunit}{kg\,m${}^{-3}$}
\newcommand{\accelunit}{m\,s${}^{-2}$}
\newcommand{\maccelunit}{m${}^{4}$\,s${}^{-2}$}
\newcommand{\dischargeunit}{m${}^3$\,s${}^{-1}$}

\newcommand{\ten}[1]{$\cdot 10^{#1}$}
\newcommand{\degrees}{${}^o$}

\newcommand{\figref}[1]{\ref{fig:#1}}
\newcommand{\Fig}{Fig.~}

\newcommand{\todo}[1]{This section still has to be written by #1}

\parindent 0cm

\newcommand{\importstr}[2]{%
\begin{figure}
\begin{alltt}
\input{#1.str}
\end{alltt}
\caption{#2}
\label{fig:#1}
\end{figure}
}

\MakeShortVerb{\|}

\hyphenation{ ba-thy-me-try stop-ped sim-u-la-ted }

%%%%%%%%%%%%%%%%%%%%%%%%%%%%%%%%%%%%%%%%%%%%%%%%%%%%%%%%% front matter

\title{%
	\shy{} 
	\\Finite Element Model for Coastal Seas
	\\~
	\\User Manual
	}

\author{%
	The SHYFEM Group
	\\Georg Umgiesser
	\\Oceanography, ISMAR-CNR
	\\Arsenale Tesa 104, Castello 2737/F
	\\30122 Venezia, Italy
	\vspace{0.5cm}
	\\georg.umgiesser@ismar.cnr.it
	\vspace{1cm}
	\\Version \VERSION
	}

%\address{ISDGM-CNR}

%\date{}		%uncomment for no date

%%%%%%%%%%%%%%%%%%%%%%%%%%%%%%%%%%%%%%%%%%%%%%%%%%%%%%%%% document

\begin{document}

\bibliographystyle{plain}

\pagenumbering{roman}
\pagestyle{plain}

\maketitle

%\begin{abstract}
% to be written
%\end{abstract}

\thispagestyle{empty}

\newpage
\tableofcontents
\newpage

%%%%%%%%%%%%%%%%%%%%%%%%%%%%%%%%%%%%%%%%%%%%%%%%%%%%%%%%%%%%%%%%%%%%%%%%
%%%%%%%%%%%%%%%%%%%%%%%%%%%%%%%%%%%%%%%%%%%%%%%%%%%%%%%%%%%%%%%%%%%%%%%%
%%%%%%%%%%%%%%%%%%%%%%%%%%%%%%%%%%%%%%%%%%%%%%%%%%%%%%%%%%%%%%%%%%%%%%%%

\chapter*{Disclaimer}
	\addcontentsline{toc}{chapter}{Disclaimer}

	\input{disclaimer}

\newpage
\pagenumbering{arabic}

%%%%%%%%%%%%%%%%%%%%%%%%%%%%%%%%%%%%%%%%%%%%%%%%%%%%%%%%%%%%%%%%%%%%%%%%
%%%%%%%%%%%%%%%%%%%%%%%%%%%%%%%%%%%%%%%%%%%%%%%%%%%%%%%%%%%%%%%%%%%%%%%%
%%%%%%%%%%%%%%%%%%%%%%%%%%%%%%%%%%%%%%%%%%%%%%%%%%%%%%%%%%%%%%%%%%%%%%%%

% external software
% contributors

\chapter{Overview}

	\section{What is it}
	\input{over_what}

	%\section{Why use it}
	%\todo{Georg}

	%\section{How to use it}
	%\todo{Georg}

	\section{Where and how to get it}
	\input{git}

	%\section{Features}
	%\todo{Georg}

	%\section{License}
	%\todo{Georg}

%%%%%%%%%%%%%%%%%%%%%%%%%%%%%%%%%%%%%%%%%%%%%%%%%%%%%%%%%%%%%%%%%%%%%%%%
%%%%%%%%%%%%%%%%%%%%%%%%%%%%%%%%%%%%%%%%%%%%%%%%%%%%%%%%%%%%%%%%%%%%%%%%
%%%%%%%%%%%%%%%%%%%%%%%%%%%%%%%%%%%%%%%%%%%%%%%%%%%%%%%%%%%%%%%%%%%%%%%%

\chapter{Installing SHYFEM}

	\section{Downloading and unpacking}
	\input{download}

	\section{Needed software}
	\input{software}

	\section{Installation}
	\input{installation}

	\section{Compilation}
	\input{compilation}

	\section{Compatibility problems}
	\input{compatibility}

	\section{Compilation options}
	\input{summary_installing}

%%%%%%%%%%%%%%%%%%%%%%%%%%%%%%%%%%%%%%%%%%%%%%%%%%%%%%%%%%%%%%%%%%%%%%%%
%%%%%%%%%%%%%%%%%%%%%%%%%%%%%%%%%%%%%%%%%%%%%%%%%%%%%%%%%%%%%%%%%%%%%%%%
%%%%%%%%%%%%%%%%%%%%%%%%%%%%%%%%%%%%%%%%%%%%%%%%%%%%%%%%%%%%%%%%%%%%%%%%

\chapter{Preprocessing: The numerical grid}

	\section{Overview}
	\input{grid_overview}

	\section{Coastline and bathymetry}
	\input{grid_convert}

	\section{Boundary line: smoothing and reducing}
	\input{grid_reduce}

%	\section{Other information}
%	\input{grid_else}

	\section{Manipulating nodes, lines and elements: the grid program}
	\input{grid_prg}

	\section{Creating the mesh}
	\input{creating_mesh}

	\section{Interpolating the bathymetry into the grd file}
	\input{bathy_interp}

	\section{Creating the bas file}
	\input{pre_basin}

%%%%%%%%%%%%%%%%%%%%%%%%%%%%%%%%%%%%%%%%%%%%%%%%%%%%%%%%%%%%%%%%%%%%%%%%
%%%%%%%%%%%%%%%%%%%%%%%%%%%%%%%%%%%%%%%%%%%%%%%%%%%%%%%%%%%%%%%%%%%%%%%%
%%%%%%%%%%%%%%%%%%%%%%%%%%%%%%%%%%%%%%%%%%%%%%%%%%%%%%%%%%%%%%%%%%%%%%%%

\chapter{Running SHYFEM}

	\input{running_shyfem}

	\section{The parameter input file (str)}
	\input{model_how_to_run}

%	\section{Adjusting to the basin}
%	\todo{Georg}

%	\section{Basic file formats}
%	\todo{Georg}

	\section{Basic usage}

		\input{basic_usage}

		\subsection{Minimal simulation}
		\input{basic_minimal}

%		\input{usage.tex}	%??

		\subsection{Boundary conditions}
		\input{bound_cond}

%		\subsection{Writing output}
%		\todo{Georg}

%		\subsection{Simulating passive tracers}
%		\todo{Georg}

%		\subsection{Temperature and salinity}
%		\todo{Georg}

%		\subsection{Linear and non-linear simulations}
%		\todo{Georg}

%		\subsection{Coriolis force}
%		\todo{Georg}

		\subsection{Wind forcing}
		\input{wind_forcing}

	\section{Advanced usage}

		\subsection{Variable time step}
		\input{var_time_step}

		\subsection{3D computations}
		
%------------------------------------------------------------------------
%
%    Copyright (C) 1985-2020  Georg Umgiesser
%
%    This file is part of SHYFEM.
%
%    SHYFEM is free software: you can redistribute it and/or modify
%    it under the terms of the GNU General Public License as published by
%    the Free Software Foundation, either version 3 of the License, or
%    (at your option) any later version.
%
%    SHYFEM is distributed in the hope that it will be useful,
%    but WITHOUT ANY WARRANTY; without even the implied warranty of
%    MERCHANTABILITY or FITNESS FOR A PARTICULAR PURPOSE. See the
%    GNU General Public License for more details.
%
%    You should have received a copy of the GNU General Public License
%    along with SHYFEM. Please see the file COPYING in the main directory.
%    If not, see <http://www.gnu.org/licenses/>.
%
%    Contributions to this file can be found below in the revision log.
%
%------------------------------------------------------------------------

\subsubsection{General information}

The basic way to run the model is in 2D, computing for each element of the
grid one value for the whole water column.  All the variables are computed
in the center of the layer, halfway down the total depth.  Deeper basins
or highly variable bathymetry can require for the correct reproduction
of the velocities, temperature and salinity the need for 3D computation.

The 3D computation is performed with $z-$layers, sigma layers or with
a hybrid formulation.  In the default $z-$layer formulation, each layer
horizontally has constant depth over the whole basin, but vertically
the layer thickness may vary between different layers. However, the
first layer (surface layer) is of varying thickness because of the water
level variation, and the last layer of an element might be only partially
present due to the bathymetry.

Layers are counted from the the surface layer (layer 1) down to the
maximum layer, depending again on the local depth. Therefore, elements
(and nodes) normally have a different total number of layers from one to
each other. This is opposed to sigma layers where the number of total
layers is constant all over the basin, but the thickness of each layer
varies between different elements.

\subsubsection{$z-$layers}

In order to use $z-$layers for 3D computations a new section |$layers|
has to be introduced into the |STR| file, where the sequence of depth
values of the bottom of the layers has to be declared.  Layer
depths must be declared in increasing order. An example of a |$layer|
section is given in figure \figref{zlayers}. Please note that the maximum
depth of the basin in the example must not exceed 20 m.

\begin{figure}[ht]
\begin{verbatim}
$layers
	2 4 6 8 10 13 16 20
$end
\end{verbatim}
\caption{Example of section {\tt \$layers} for z layers. The maximum depth of 
the basin is 20 meters. The first 5 layers have constant thickness 
of 2 m, while the last three vary between 3 and 4 m.}
\label{fig:zlayers}
\end{figure}

A specific treatment for the bottom layer has to be carried out.  In fact,
if the model runs on basins with variable bathymetry, for each element
there will be a different total number of layers. The bathymetric value
normally does not coincide with one of the layer depths, and therefore
the last layer must be treated separately.

To declare how to treat the last layer two parameters have to be
inserted in the |$para| section. The first is |hlvmin|, the minimum
depth, expressed as a percentage with respect to the full layer depth,
ranging between 0 and 1, This is the fraction that the last layer
must have in order to be maintained as a separate layer.  The second
parameter is |ilytyp| and it defines the kind of adjustment done on the
last layer. If it is set to 0 no adjustment is done, if it is set to 1
the depth of the last layer is adjusted to the one declared in the |STR|
file (full layer change).  If it is 2 the adjustment to the previous layer
is done only if the fraction of the last layer is smaller than |hlvmin|
(change of depth).  If it is 3 (default) the bathymetric depth is kept
and added to the last but one layer. Therefore with a value of 0 or 3
the total depth will never be changed, whereas with the other levels
the total depth might be adjusted.

As an example, take the layer definition of \Fig\figref{zlayers}. Let |hlvmin|
be set to 0.5, and let an element have a depth of 6.5 m. The total number
of layers is 4, where the first 3 have each a thickness of 2 m and the
last layer of this element (layer 4) is 0.5 m. However, the nominal
thickness of layer 4 is 2 m and therefore its relative thickness is 0.25
which is smaller than |hlvmin|. With |ilytyp|=0 no adjustment will be
done and the total number of layers in this element will be 4 and the
last layer will have a thickness of 0.5 m.  With |ilytyp|=1 the total
number of layers will be changed to 3 (all of them with 2 m thickness)
and the total depth will be adjusted to 6 m. The same will happen with
|ilytyp|=2, because the relative thickness in layer 4 is smaller than
|hlvmin|.  Finally, with |ilytyp|=3 the total number of layers will be
changed to 3 but the remaining depth of 0.5 m will be added to layer 3
that will become 2.5 m.

In the case the element has a depth of 7.5 m, the relative thickness is
now 0.75 and greater than |hlvmin|.  In this case, with |ilytyp|=0, 2
and 3 no adjustment will be done and the total number of layers in this
element will be 4 and the last layer will have a thickness of 1.5 m.
With |ilytyp|=1 the total number of layers will be kept as 4 but the
total depth will be adjusted to 8 m. This will make all layers equal to
2 m thickness.

A specific treatment for the surface layer is also needed. Standard 
$z-$layers are coded with the first layer of variable thickness that, 
however, must never become dry. This is the default usage of $z-$layers.
As an alternative, the $z$-star layers can be used. You need to specify in the 
|$para| section the parameter |nzadapt| equal or greater to
the maximum number of layers. For the previous example
one should set |nzadapt|$\ge 8$. If one wants to use fixed
interface for the interior part of the water column, there is also the
possibility to move only the surface layers with a $z-$star type 
transformation. This reminds of $z$-star over $z-$layers. To use
this slicing, you need to set |nzadapt|, the minimum number of moving layers 
(when the water level moves downward). For example |nzadapt|$=2$ means that, 
at minimum, two layers will absorb a downward motion of the water level. 
Please note that this feature is still experimental.


\subsubsection{Sigma layers}

Sigma layers use a constant number of layers all over the basin. They
are easier to use than z layers, because only one parameter has to
be specified. In the |$para| section of the |STR| file 
the parameter |nsigma| has to be set
to the number of desired sigma layers. The layers are then equally spaced
between each other.

Sigma layers can be also specified in the |$layers| section. In this case
the negative percentage of the layers have to be given.
An example is given in figure \figref{slayers}. This is only useful if
the layers are not equally spaced, because for equally spaced sigma layers
the parameter |nsigma| in the |STR| can be used.

In the bathymetry file depth values have to be given on nodes and not
on elements. in case the utility |shybas| can be used to convert between
elemental and nodal depth values.

\begin{figure}[ht]
\begin{verbatim}
$layers
	-0.2 -0.4 -0.6 -0.8 -1.0
$end
\end{verbatim}
\caption{Example of section {\tt \$layers} for sigma layers.
The depth is divided in 5 equally spaced layers. Please note that
this division could have also been achieved setting {\tt nsigma} to the
value of 5.}
\label{fig:slayers}
\end{figure}

\subsubsection{Hybrid layers}

Hybrid layers are also called "sigma over zeta" layers. They are a
mixture between sigma layers close to the surface and zeta layers in
the deeper parts of the basin. This is useful if strong bathymetry
gradients are present. This avoids possible instabilities due to the
sigma layers in the deeper parts.

For the hybrid layers a depth of closure has to be defined. This is the
depth value above which sigma layers are used and below which zeta levels
are applied. Please note that the basin has to be prepared in order that
depth values are given on nodes and in an elements the three depth values
on the vertices are either higher equal or lower equal than the depth
of closure. The routine |shybas| can be used in order to create
a grid compatible with hybrid coordinates. An example of how
to specify hybrid layers is given in
figure \figref{hlayers}.

\begin{figure}[ht]
\begin{verbatim}
$layers
	-0.2 -0.4 -0.6 -0.8 10. 20. 30. 40. 50
$end
\end{verbatim}
\caption{Example of section {\tt \$layers} for hybrid layers.
The depth is divided in 5 equally sigma layers on the surface above 10 meters
(which is the depth of closure) and zeta layers below until a depth
of 50 meters.}
\label{fig:hlayers}
\end{figure}

Please note that hybrid layers are still experimental. So use with care.




\subsubsection{Vertical viscosity}

The introduction of layers requires also to define the values of
vertical eddy viscosity and eddy diffusivity.  In any case a value of
these two parameters has to be set if the 3D run is performed. This
could be done by setting a constant value of the parameters |vistur|
(vertical viscosity) and |diftur| (vertical diffusivity). In this case
possible values are between 1\ten{-2} and 1\ten{-5}, depending on the
stability of the water column. Higher values (1\ten{-2}) indicate higher
stability and a stronger barotropic behavior.

The other possibility is to compute the vertical eddy coefficients through
a turbulence closure scheme. This usage will be described in the section
on turbulence.



		\subsection{Baroclinic terms}
		\input{barocl}

		\subsection{Restart}
		\input{restart}

		\subsection{Turbulence}
		\input{turbulence}

		\subsection{Sediment transport}
		\input{sediment}

		\subsection{Coupling with waves}
		\input{S_wave}

		\subsubsection{WAVEWATCH III model}
		
%------------------------------------------------------------------------
%
%    Copyright (C) 1985-2020  Georg Umgiesser
%
%    This file is part of SHYFEM.
%
%    SHYFEM is free software: you can redistribute it and/or modify
%    it under the terms of the GNU General Public License as published by
%    the Free Software Foundation, either version 3 of the License, or
%    (at your option) any later version.
%
%    SHYFEM is distributed in the hope that it will be useful,
%    but WITHOUT ANY WARRANTY; without even the implied warranty of
%    MERCHANTABILITY or FITNESS FOR A PARTICULAR PURPOSE. See the
%    GNU General Public License for more details.
%
%    You should have received a copy of the GNU General Public License
%    along with SHYFEM. Please see the file COPYING in the main directory.
%    If not, see <http://www.gnu.org/licenses/>.
%
%    Contributions to this file can be found below in the revision log.
%
%------------------------------------------------------------------------

SHYFEM has been coupled to the WAVEWATCH III (WW3) wave model.
The SHYFEM model was coupled to WW3 based on a so called ``hard coupling'', e.g.
binding the models directly without any so called coupling library such as
(ESMF, OASIS, PGMCL or others). The benefit is on the hand, minimal memory
usage, very limited code changes in both models. Top-level approach using the
coupled model framework based on having a routine for initialization,
computation and finalization, that allows a neat inclusion of WW3 in any kind
of flow model.

\paragraph{Installation and compilation}
%The coupled model is disributed within the SHYFEM distribution and the WW3 code
%is located in the |shyfem/WW3| subfolder. The latest version of the WW3 code is
%available on GitHub at \url{https://github.com/NOAA-EMC/WW3}.

In order to compile the coupled SHYFEM-WW3 model, the variable |WW3| in the 
file |Rules.make| should be set as |WW3 = true| and |WW3DIR| should
point to the folder containing the |WW3| code (e.g., WW3DIR = \$(HOME)/bin/WW3). 
|SHYFEM| should be compiled for running in parallel by setting 
|PARALLEL_MPI = NODE|.

Moreover, WW3 also needs additional libraries:
\begin{itemize}
\item |netcdf| compiled with the same compiler.  You must set |NETCDF = true|
and specify the variable |NETCDFDIR| to indicate the directory where the 
libraries and its include files can be found.
\item |METIS| and |PARMETIS| compiled with the same compiler. You must set
|PARTS = PARMETIS| and set the variables |METISDIR| and |PARMETISDIR|
to indicate the directory where the libraries and its include files can be
found.
\end{itemize}

The compilation of |WW3| can be customized changing a set of model options,
defined with the variable |WW3CFLAGS| in the file |fem3d/Makefile|. They
correspond to the parameters listed in the file |switch| needed by the stand 
alone WW3 installation. For a detailed description of the options see the WW3 
manual (section 5.9).

You can now follow the general |SHYFEM| installation instruction to compile the 
code.

It is however required to download, compile and install the stand alone |WW3| 
model from the ERDC GitHub at |https://github.com/erdc/WW3/tree/ww3_shyfem|. 
This step is needed for having |WW3| source code and pre- and post-processing 
tools (e.g., ww3\_grid). 
It is important that the so called ``switch'' file of |WW3| (see |WW3| documentation) 
contains the same parameters listed with variable |WW3CFLAGS| are identical 
for both modules. The default setup provided with |WW3CFLAGS| is basically 
identical to the setup of SHOM and Meteo France as well as the USACE based 
on implicit time stepping on unstructured grids. Basically, there is no need 
for modification of the ``switch'' file with the given settings but all 
settings are supported as long the unstructured grid option in WW3 is used.

For installing and comping the stand alone |WW3| model you can follow
the following procedure (see the |WW3| manual for more informations):
\begin{enumerate}
\item download the ww3\_shyfem branch from the ERDC git repository: 
git clone --branch ww3\_shyfem https://github.com/erdc/WW3/tree/ww3\_shyfem;
\item move to the WW3 directory;
\item set the NetCDF path: with |export WWATCH3_NETCDF=NC4| and \\
|export NETCDF_CONFIG=/netcdf-dir/nc-config|;
\item setup the model using |./model/bin/w3_setup /home/model/bin/WW3/model -c| \\
| <cmplr> -s <swtch>| with |cmplr| the compiler |comp| and |link| files (e.g., shyfem
for files comp.shyfem and link.shyfem) and |swtch| the switch file (e.g., shyfem 
for a switch file located in ./model/bin having name switch\_shyfem).
\item compile |WW3| with |w3_automake|, or for a few programs with
|./model/bin/w3_automake| \\ |ww3_grid ww3_shel ww3_ounf| with name 
\item the compiled programms are located in |./model/exe/ww3_grid|
\end{enumerate}

\paragraph{Coupling description}
The implementation was done by developing two modules, one in
|SHYFEM| (subww3.f) and the other one in |WW3| (w3cplshymfen.F90). The 
SHYFEM coupling module is connected by the ``use'' statement to the 
modules of the |WW3| code. In this way one can access all fields from 
the |WW3| in |SHYFEM| and vice versa. In the |WW3| module all spectral based
quantities are computed and |SHYFEM| is directly assigning the various 
arrays, based on global arrays, to the certain domains for each of the 
model decompositions.

In terms of |WW3| we have used the highest-level implementation based
on the |WW3| multigrid driver. This allows basically to couple any kind
of grid type with |SHYFEM| (rectangular, curvi-linear, SMC and
unstructured). It offers of course the possibility to run |WW3| based
on a multigrid SETUP and coupled to |SHYFEM|. Both models use their
native input files for running the model except that the forcing by
wind within the coupled model is done via |SHYFEM| and the flow field
is by definition provided based on the coupling to |SHYFEM|. In this
way both models are fully compatible to the available documentation.

The two numerical models (SHYFEM and WW3) should exchange all the
variables that are needed to simulate the current-wave interaction.
The following variables sare passed by |SHYFEM| to |WW3|:
\begin{itemize}
\item water levels;
\item three-dimensional water currents;
\item wind components.
\end{itemize}

The following physical quantities have been computed in |WW3| and are 
available for |SHYFEM|:
\begin{itemize}
\item significant wave height;
\item mean wave period;
\item peak wave period;
\item main wave direction (the where the wave go);
\item radiation stress,
\item wave pressure;
\item wind drag coefficient;
\item Stokes velocities.
\end{itemize}

The exchange module of SHYFEM contains all the needed infrastructure
for initializing, calling and finalizing the wave model run. Two
subroutines (getvarSHYFEM and getvarWW3), which are called before and
after the call to the wave model in order to obtain currents, water
levels and on the other hand integral wave parameters and radiation
stresses.

The significant wave height, mean wave period and main wave 
direction are written by |SHYFEM| in the |.wave.shy| file according
to the values of the variables |idtwav| and |itmwav| in the |wave|
section of the |SHYFEM| parameter file (see below).

\paragraph{Running the coupled SHYFEM-WW3 model}
Since the wave output are written by |SHYFEM|, the output are
set to off in |WW3|, which reduces disk usage and significantly reduces 
the parallel overhead due to the output part. 

As we have utilized here the implicit scheme in WW3, which was
developed by Roland \& Partner and the implicit scheme is well
validated and unconditionally stable. The choice of the time step
should be according to the physical time scales of the modelled
processes. 

In order to run the coupled model, we suggest the following
procedure:
\begin{enumerate}
\item Setup the |SHYFEM| set-up for the region of interest. 
|SHYFEM| needs a mesh in the .bas format, a parameter file and 
all forcing files (see |SHYFEM| documentation). In the |SHYFEM| 
parameter file the section |\$waves| must be activated with |iwave = 11|.
The time step for coupling with WW3 |dtwave| must be set equal
to the |SHYFEM| and |WW3| timesteps (TO BE FIXED).
|idtwav|, |itmwav| should also be set for determining the 
time step and start time for writing to the output file |.wave.shy| .
Preprocess the |SHYFEM| grid with a selected number of domains 
without bandwidth optimization with |shypre -noopti grid.grd|.

\item Setup the |WW3| model for the region of interest. In the
running directoty, |WW3| requires:
\begin{enumerate}
\item the file |ww3_grid.nml| containing the spectrum, run, timesteps
and grid parameterizations (see |WW3| documentation). This file also
contains the names of mesh and namelist files.
\item the namelist file containing the variables for setting
the tunable parameters for source terms, propagation schemes, and 
numerics.
\item the file |ww3_multi.nml| which handles the time steps and 
the I/O. Since |WW3| receives flow and wind from |SHYFEM| there are 
no input fields that need to be prescribed. The output part of the 
wave model itself is handled by |SHYFEM| as well.
\item the numerical mesh in the GMSH format .msh. The |SHYFEM|
mesh in .grd format can be easily converted to .msh with the command:
shybas -msh grid.bas (it creates a bas.msh file).
\end{enumerate}

Before running the model, the setup |ww3_grid| tool (found in the
stand alone |WW3/build/bin| directory) needs to be run and the output, 
which is named per default ``mod\_def.ww3'' needs to be renamed 
with the extension defined in parameter |MODEL(1)\%NAME = 'med'|
in |ww3_multi.nml| (e.g., mod\_def.med in the above mentioned case). 
The |ww3_grid| tool must be run every time the namelist or grid files 
are modified. 

\item run the coupled model with the |shyfem| command. Since the code
is optimized to run with MPI on domain decomposition, we suggest to
run the coupled model in parallel using |mpiexec -np np shyfem namelist.str| 
with np the number of processors.

\end{enumerate}



		\subsection{Tidal potential}
		\input{S_tide}

		\subsection{Meteo forcing}
		\input{meteo_forcing}

		\subsection{Lagrangian particle module}
		Lagrangian analysis provides a powerful tool to evaluate the output
of ocean circulation models. SHYFEM is equipped with a 3-D 
particle-tracking module, which simulates the trajectory of 
particles as a function of the hydrodynamics. 

The vertical components of the turbulent diffusion velocity is
computed using the Milstein scheme reported by Grawe 
\cite{Grawe2010}. The horizontal diffusion was computed using a 
random walk technique based on Fisher \cite{Fisher1979}, with the 
turbulent diffusion coefficients obtained by means of the 
Smagorinsky \cite{Smagorinsky1993} formulation. The wind drag 
and Stokes drift contribution to the total transport is parametrized
by |stkpar| factor. An additional calibration parameter to account
for the drifter inertia could be set (|dripar|). The model allows
particle to beach on the shore (|lbeach|). 

The particle-tracking model can be also used off-line (parameter
|idtoff|). In this case it uses the Eulerian hydrodynamic fields 
generated by the forecast system. The main advantage of the 
off-line approach is that the trajectory calculation typically 
takes much less computational effort than the driving hydrodynamic 
model.

The lagrangian particles can be released:
\begin{itemize}
\item inside the given areas (filename |lgrlin|). If this file is not 
      specified they are released over the whole domain. The amount of
      particles released and the time step is specified by |nbdy| and 
      |idtl|.
\item at selected times and location, e.g. along a drifter track
      (filename |lgrtrj|). |nbdy| particles are released at the times
      and location specified in the file.
\item as initial particle distribution (filename |lgrini|) at time
      |itlgin|. This file has the same format as the lagrangian output.
\item at the open boundaries, either as particles per second or per
      volume flux (parameter |lgrpps|).
\end{itemize}

The particle-tracking model is activated by setting |ilagr| > 0.
The lagrangian module runs between the times |itlanf| and |itlend|.
See more details in the list of parameters and they description 
reported in the appendix.

The lagrangian model can be used to specifically simulate sediments
(|ised=1|), oil (|ioil=1|) and larvae (|ilarv=1|).


%		\subsection{Large grids and projections}
%		\todo{Debora}

%%%%%%%%%%%%%%%%%%%%%%%%%%%%%%%%%%%%%%%%%%%%%%%%%%%%%%%%%%%%%%%%%%%%%%%%
%%%%%%%%%%%%%%%%%%%%%%%%%%%%%%%%%%%%%%%%%%%%%%%%%%%%%%%%%%%%%%%%%%%%%%%%
%%%%%%%%%%%%%%%%%%%%%%%%%%%%%%%%%%%%%%%%%%%%%%%%%%%%%%%%%%%%%%%%%%%%%%%%

%\chapter{Other modules}

%	\section{Residence times}
%	\todo{Andrea}

%	\section{Ecological module (EUTRO)}
%	\todo{Michol (c'e' gia' una descrizione di Donata)}

%	\input{bio3d.tex}
%	\subsection{Parameters for the Water Quality Module}
%	\input{S_biopar_h.tex}

%%%%%%%%%%%%%%%%%%%%%%%%%%%%%%%%%%%%%%%%%%%%%%%%%%%%%%%%%%%%%%%%%%%%%%%%
%%%%%%%%%%%%%%%%%%%%%%%%%%%%%%%%%%%%%%%%%%%%%%%%%%%%%%%%%%%%%%%%%%%%%%%%
%%%%%%%%%%%%%%%%%%%%%%%%%%%%%%%%%%%%%%%%%%%%%%%%%%%%%%%%%%%%%%%%%%%%%%%%

%\chapter{Postprocessing}
%
% Da qui in poi ci pensiamo piu' tardi...
%
%\section{Running the post processing routines}
%\section{Time series: gnuplot}
%\section{Plotting spatial data: plots}
%	\subsection{Basic usage}
%	\subsection{Advanced usage}
%\section{Exporting data}

%%%%%%%%%%%%%%%%%%%%%%%%%%%%%%%%%%%%%%%%%%%%%%%%%%%%%%%%%%%%%%%%%%%%%%%%
%%%%%%%%%%%%%%%%%%%%%%%%%%%%%%%%%%%%%%%%%%%%%%%%%%%%%%%%%%%%%%%%%%%%%%%%
%%%%%%%%%%%%%%%%%%%%%%%%%%%%%%%%%%%%%%%%%%%%%%%%%%%%%%%%%%%%%%%%%%%%%%%%

%\chapter{Final thoughts}

%%%%%%%%%%%%%%%%%%%%%%%%%%%%%%%%%%%%%%%%%%%%%%%%%%%%%%%%%%%%%%%%%%%%%%%%
%%%%%%%%%%%%%%%%%%%%%%%%%%%%%%%%%%%%%%%%%%%%%%%%%%%%%%%%%%%%%%%%%%%%%%%%
%%%%%%%%%%%%%%%%%%%%%%%%%%%%%%%%%%%%%%%%%%%%%%%%%%%%%%%%%%%%%%%%%%%%%%%%
%%%%%%%%%%%%%%%%%%%%%%%%%%%%%%%%%%%%%%%%%%%%%%%%%%%%%%%%%%%%%%%%%%%%%%%%
%%%%%%%%%%%%%%%%%%%%%%%%%%%%%%%%%%%%%%%%%%%%%%%%%%%%%%%%%%%%%%%%%%%%%%%%



\appendix


\chapter{Hydrodynamic equations and resolution techniques}

	\input{equat.tex}

\chapter{File formats}
\label{file_formats}

%	\section{STR file}

	\section{GRD file}
	\input{grd_format}

%	\section{Time series}

%	\section{Output file formats}





\chapter{Parameter list}


\section{Parameter list for the SHYFEM model}


\subsection{Section {\tt \$title}}

This section must be always the first section in the parameter input file.
It contains only three lines. An example is given in 
figure \ref{fig:titleexample}.

\begin{figure}[ht]
\begin{verbatim}
$title
        free one line description of simulation
        name_of_simulation
        name_of_basin
$end
\end{verbatim}
\caption{Example of section {\tt \$title}}
\label{fig:titleexample}
\end{figure}

The first line of this section is a free one line description of
the simulation that is to be carried out. The next line contains
the name of the simulation.
All created files will use this name in the main part of the file name
with different extensions. Therefore the hydrodynamic output file
(extension |out|) will be named |name_of_simulation.out|.
The last line gives the name of the basin file to be used. This
is the pre-processed file of the basin with extension |bas|.
In our example the basin file |name_of_basin.bas| is used.

The directory where this files are read from or written to depends
on the settings in section {\tt \$name}. Using the default
the program will read from and write to the current directory.

\subsection{Section {\tt \$para}}
\input{S_para_h.tex}

\subsection{Section {\tt \$proj}}
\input{P_proj.tex}

\subsection{Section {\tt \$waves}}
\input{P_waves.tex}

\subsection{Section {\tt \$sedtr}}
\input{P_sediment.tex}

\subsection{Section {\tt \$wrt}}
\input{P_wrt.tex}

\subsection{Section {\tt \$lagrg}}
\input{P_lagrg.tex}

\subsection{Section {\tt \$name}}
\input{S_name.tex}

\subsection{Section {\tt \$bound}}
\input{S_bound.tex}

\subsection{Section {\tt \$wind}}
\input{S_wind.tex}


\subsection{Section {\tt \$extra}}

In this section the node numbers of so called ``extra'' points are given. 
These are points where the value of simulated variables (water level, 
velocities, temperature, salinity, tracer, etc.) are written to create
a time series that can be elaborated later. The output for these ``extra''
points consumes little memory and can be therefore written with a
much higher frequency (typically the same as the integration time step)
than the complete hydrodynamic output. The output is written
to file EXT.

The format of the section is the following:
\begin{verbatim}
$extra
	node1   'string1'
	node2   'string2'
	etc..
$end
\end{verbatim}
where node is the node number and string is a description of the node.
If no description strings are needed the nodes can also be specified
by just giving their values:
\begin{verbatim}
$extra
	node1  node2  node3
	node4  etc..
$end
\end{verbatim}
This format is however deprecated.


\subsection{Section {\tt \$flux}}

In this section transects are specified through which the discharge
of water is computed by the program and written to file FLX.
The transects are defined by their nodes through which they run.
All nodes in one transect must be adjacent, i.e., they must form a
continuous line in the FEM network.

The nodes of the transects are specified in free format and are
ended with the description of the section.
An example is given here:
\begin{verbatim}
$flux
	1001 1002 1004 'section 1'
	35 37 46 'special section'
	407
	301 'section given on two lines'
$end
\end{verbatim}
The example shows the definition of 3 transects. As can be seen, the 
nodes of the transects can be given on one line alone (first transect),
or on more than one lines (transect 3).
There is also an old format that seperates one section from the other by
inserting the value 0. However, this format is deprecated.


\section{Parameter list for the post processing routines}

The format of the parameter input file is the same as the one for
the main routine. Please see this section for more information
on the format of the parameter input file.

Some sections of the parameter input file are identical to the 
sections used in the main routine. For easier reference we will
repeat the possible parameters of these section here.


\subsection{Section {\tt \$title}}

This section must be always the first section in the parameter input file.
It contains only three lines. An example has been given already in 
figure \ref{fig:titleexample}.

The only difference with respect to the {\tt \$para} section of the main
routine is the first line. Here any description of the output can be used.
It is just a way to label the parameter file.  The other two line with
the name of simulation and the basin are used to open the files needed
for plotting.


\subsection{Section {\tt \$para}}
\input{S_para_a.tex}

\subsection{Section {\tt \$color}}
\input{S_color.tex}

\subsection{Section {\tt \$arrow}}
\input{S_arrow.tex}

\subsection{Section {\tt \$legend}}
\input{S_legend.tex}

\subsection{Section {\tt \$legvar}}
\input{S_legvar.tex}

\subsection{Section {\tt \$name}}
\input{S_name.tex}


\bibliography{abbrev,lag,sedi,lagra}
\addcontentsline{toc}{chapter}{Bibliography}



\end{document}




