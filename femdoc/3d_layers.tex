
%------------------------------------------------------------------------
%
%    Copyright (C) 1985-2020  Georg Umgiesser
%
%    This file is part of SHYFEM.
%
%    SHYFEM is free software: you can redistribute it and/or modify
%    it under the terms of the GNU General Public License as published by
%    the Free Software Foundation, either version 3 of the License, or
%    (at your option) any later version.
%
%    SHYFEM is distributed in the hope that it will be useful,
%    but WITHOUT ANY WARRANTY; without even the implied warranty of
%    MERCHANTABILITY or FITNESS FOR A PARTICULAR PURPOSE. See the
%    GNU General Public License for more details.
%
%    You should have received a copy of the GNU General Public License
%    along with SHYFEM. Please see the file COPYING in the main directory.
%    If not, see <http://www.gnu.org/licenses/>.
%
%    Contributions to this file can be found below in the revision log.
%
%------------------------------------------------------------------------

\subsubsection{General information}

The basic way to run the model is in 2D, computing for each element of the
grid one value for the whole water column.  All the variables are computed
in the center of the layer, halfway down the total depth.  Deeper basins
or highly variable bathymetry can require for the correct reproduction
of the velocities, temperature and salinity the need for 3D computation.

The 3D computation is performed with $z-$layers, sigma layers or with
a hybrid formulation.  In the default $z-$layer formulation, each layer
horizontally has constant depth over the whole basin, but vertically
the layer thickness may vary between different layers. However, the
first layer (surface layer) is of varying thickness because of the water
level variation, and the last layer of an element might be only partially
present due to the bathymetry.

Layers are counted from the the surface layer (layer 1) down to the
maximum layer, depending again on the local depth. Therefore, elements
(and nodes) normally have a different total number of layers from one to
each other. This is opposed to sigma layers where the number of total
layers is constant all over the basin, but the thickness of each layer
varies between different elements.

\subsubsection{$z-$layers}

In order to use $z-$layers for 3D computations a new section |$layers|
has to be introduced into the |STR| file, where the sequence of depth
values of the bottom of the layers has to be declared.  Layer
depths must be declared in increasing order. An example of a |$layer|
section is given in figure \figref{zlayers}. Please note that the maximum
depth of the basin in the example must not exceed 20 m.

\begin{figure}[ht]
\begin{verbatim}
$layers
	2 4 6 8 10 13 16 20
$end
\end{verbatim}
\caption{Example of section {\tt \$layers} for z layers. The maximum depth of 
the basin is 20 meters. The first 5 layers have constant thickness 
of 2 m, while the last three vary between 3 and 4 m.}
\label{fig:zlayers}
\end{figure}

A specific treatment for the bottom layer has to be carried out.  In fact,
if the model runs on basins with variable bathymetry, for each element
there will be a different total number of layers. The bathymetric value
normally does not coincide with one of the layer depths, and therefore
the last layer must be treated separately.

To declare how to treat the last layer two parameters have to be
inserted in the |$para| section. The first is |hlvmin|, the minimum
depth, expressed as a percentage with respect to the full layer depth,
ranging between 0 and 1, This is the fraction that the last layer
must have in order to be maintained as a separate layer.  The second
parameter is |ilytyp| and it defines the kind of adjustment done on the
last layer. If it is set to 0 no adjustment is done, if it is set to 1
the depth of the last layer is adjusted to the one declared in the |STR|
file (full layer change).  If it is 2 the adjustment to the previous layer
is done only if the fraction of the last layer is smaller than |hlvmin|
(change of depth).  If it is 3 (default) the bathymetric depth is kept
and added to the last but one layer. Therefore with a value of 0 or 3
the total depth will never be changed, whereas with the other levels
the total depth might be adjusted.

As an example, take the layer definition of \Fig\figref{zlayers}. Let |hlvmin|
be set to 0.5, and let an element have a depth of 6.5 m. The total number
of layers is 4, where the first 3 have each a thickness of 2 m and the
last layer of this element (layer 4) is 0.5 m. However, the nominal
thickness of layer 4 is 2 m and therefore its relative thickness is 0.25
which is smaller than |hlvmin|. With |ilytyp|=0 no adjustment will be
done and the total number of layers in this element will be 4 and the
last layer will have a thickness of 0.5 m.  With |ilytyp|=1 the total
number of layers will be changed to 3 (all of them with 2 m thickness)
and the total depth will be adjusted to 6 m. The same will happen with
|ilytyp|=2, because the relative thickness in layer 4 is smaller than
|hlvmin|.  Finally, with |ilytyp|=3 the total number of layers will be
changed to 3 but the remaining depth of 0.5 m will be added to layer 3
that will become 2.5 m.

In the case the element has a depth of 7.5 m, the relative thickness is
now 0.75 and greater than |hlvmin|.  In this case, with |ilytyp|=0, 2
and 3 no adjustment will be done and the total number of layers in this
element will be 4 and the last layer will have a thickness of 1.5 m.
With |ilytyp|=1 the total number of layers will be kept as 4 but the
total depth will be adjusted to 8 m. This will make all layers equal to
2 m thickness.

A specific treatment for the surface layer is also needed. Standard 
$z-$layers are coded with the first layer of variable thickness that, 
however, must never become dry. This is the default usage of $z-$layers.
As an alternative, the $z$-star layers can be used. You need to specify in the 
|$para| section the parameter |nzadapt| equal or greater to
the maximum number of layers. For the previous example
one should set |nzadapt|$\ge 8$. If one wants to use fixed
interface for the interior part of the water column, there is also the
possibility to move only the surface layers with a $z-$star type 
transformation. This reminds of $z$-star over $z-$layers. To use
this slicing, you need to set |nzadapt|, the minimum number of moving layers 
(when the water level moves downward). For example |nzadapt|$=2$ means that, 
at minimum, two layers will absorb a downward motion of the water level. 
Please note that this feature is still experimental.


\subsubsection{Sigma layers}

Sigma layers use a constant number of layers all over the basin. They
are easier to use than z layers, because only one parameter has to
be specified. In the |$para| section of the |STR| file 
the parameter |nsigma| has to be set
to the number of desired sigma layers. The layers are then equally spaced
between each other.

Sigma layers can be also specified in the |$layers| section. In this case
the negative percentage of the layers have to be given.
An example is given in figure \figref{slayers}. This is only useful if
the layers are not equally spaced, because for equally spaced sigma layers
the parameter |nsigma| in the |STR| can be used.

In the bathymetry file depth values have to be given on nodes and not
on elements. in case the utility |shybas| can be used to convert between
elemental and nodal depth values.

\begin{figure}[ht]
\begin{verbatim}
$layers
	-0.2 -0.4 -0.6 -0.8 -1.0
$end
\end{verbatim}
\caption{Example of section {\tt \$layers} for sigma layers.
The depth is divided in 5 equally spaced layers. Please note that
this division could have also been achieved setting {\tt nsigma} to the
value of 5.}
\label{fig:slayers}
\end{figure}

\subsubsection{Hybrid layers}

Hybrid layers are also called "sigma over zeta" layers. They are a
mixture between sigma layers close to the surface and zeta layers in
the deeper parts of the basin. This is useful if strong bathymetry
gradients are present. This avoids possible instabilities due to the
sigma layers in the deeper parts.

For the hybrid layers a depth of closure has to be defined. This is the
depth value above which sigma layers are used and below which zeta levels
are applied. Please note that the basin has to be prepared in order that
depth values are given on nodes and in an elements the three depth values
on the vertices are either higher equal or lower equal than the depth
of closure. The routine |shybas| can be used in order to create
a grid compatible with hybrid coordinates. An example of how
to specify hybrid layers is given in
figure \figref{hlayers}.

\begin{figure}[ht]
\begin{verbatim}
$layers
	-0.2 -0.4 -0.6 -0.8 10. 20. 30. 40. 50
$end
\end{verbatim}
\caption{Example of section {\tt \$layers} for hybrid layers.
The depth is divided in 5 equally sigma layers on the surface above 10 meters
(which is the depth of closure) and zeta layers below until a depth
of 50 meters.}
\label{fig:hlayers}
\end{figure}

Please note that hybrid layers are still experimental. So use with care.




\subsubsection{Vertical viscosity}

The introduction of layers requires also to define the values of
vertical eddy viscosity and eddy diffusivity.  In any case a value of
these two parameters has to be set if the 3D run is performed. This
could be done by setting a constant value of the parameters |vistur|
(vertical viscosity) and |diftur| (vertical diffusivity). In this case
possible values are between 1\ten{-2} and 1\ten{-5}, depending on the
stability of the water column. Higher values (1\ten{-2}) indicate higher
stability and a stronger barotropic behavior.

The other possibility is to compute the vertical eddy coefficients through
a turbulence closure scheme. This usage will be described in the section
on turbulence.

